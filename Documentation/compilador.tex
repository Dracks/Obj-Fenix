\chapter {Compilador}
\section {Funcionamiento}
La primera intencion del compilador es que se haga mediante flex, bison y symtab 
en C, aunque seria interesante que mirasemos de hacerlo en C++, por eso de la orientacion a objetos.

Para facilitar el trabajo al usuario para compilar un programa con multiples archivos
nuestro compilador debera ser listo y compilar tambien los archivos que necesite que
requieran compilacion.

Analizando el problema de no tener cabeceras, hemos llegado a la conclusion que el
analisis sintactico (bison) y el analisis semantico (symtab, c, tipos de elementos,
bla bla bla...), se debera realizar de forma separada, permitiendo asi al bison, 
empezar a cargar la semantica base en la tabla de simbolos, y despues hacer el
analisis semantico y generar el codigo intermedio en otro punto.

Esto implica que debermos tener una lista de clases a analizar semanticamente.

\subsection {Analisis sintactico, bison}
Este analisis a parte de comprovar que sintacticamente el codigo es correcto, se
encargara de generar una lista de clases, con todas sus variables, sus metodos y sus implementaciones.

Utilizando un doble identificador de los elementos, permitiremos analizar sintacticamente,
y despues passarlo al analisis semantico.

En este es donde deberemos identificar que tenemos un import, y cargar el bytecode de este,
o a�adirlo a una cola de compilacion de debera tener el compilador, para passar por todos los archivos.
Este punto es el que debemos tener claro si queremos que nos compile paquete a paquete
y una los bytecodes, o nos genere un solo bytecode por defecto.


\subsection {Analisis semantico}
Este recorrera la lista de classes que nos habra generado en analisis sintactico,
de todo el codigo, y a medida que trate, nos ira generando un codigo intermedio
(aqui es donde entran en juego los objetos, si utilizamos objetos, podemos cambiar
el objeto de generar codigo intermedio para cambiar el codigo intermedio resultante).

\subsection {Compilacion}

Esto sera un objeto el cual nos ira generando el archivo de codigo resultante,
pudiendo ser este un bytecode propio, un bytecode externo o llvm. 

\section {Parametros}
Por defecto el compilador utilizara un listado de m�dulos por defecto, que estos
mediante parametros se podrian reemplazar para utilizar otras librer�as. Por
Ejemplo, por defecto utilizaremos las SDL, con gr�ficos 2D, pero si alguien
implementa la correspondiente interfaz sobre OpenGL, que le pudieras indicar que
utilizara la interfaz OGL.

