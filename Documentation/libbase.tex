\chapter{Librerias}

Aqu\'i definiremos las librerias base que debera tener el compilador, y las extras que podemos a�adir al compilador mediante parametros.

\section{Libreria Base}
Esta libreria es la que se cargara con el mismo compilador y optendremos los datos base.

Para empezar tendremos:
\begin{enumerate}
\item{Integer: objeto entero (para definir)}
\item{Boolean: objeto Booleano (para definir)}
\item{Float: objeto real (para definir)}
\\
\item{Punto2DE: Punto para las 2 Dimensiones entero}
\item{Punto2DF: Punto para las 2 Dimensiones real}
\item{Punto3DE: Punto para las 3 Dimensiones entero}
\item{Punto3DF: Punto para las 3 Dimensiones real}
\\
\item{List: Clase nativa de implementaci�n de las listas.} 
\\
\item{File: Acceso a archivos para guardar o leer datos, siendo estos configuraciones o lo que sea.}
\\
\item{Shell: Clase para acceder a la consola, siendo lectura o escriptura de datos (creo que deberia heredar de file o a lo mejor tener clientela con el)}
\end{enumerate}

\section{Librerias Extras}
Podriamos empezar por algo simple, como seria esto:
\begin{enumerate}
\item{Game (o Game2D): Libreria Input-output por interficie grafica (defecto).}
\item{Sound: Libreria... OpenAL? por ejemplo, para sonidos.}
\item{Game3D o OpenGL: Implementacion de la openGL para obj-Fenix, (no lo haremos en principio) (excluye a Game);}
\end{enumerate}
